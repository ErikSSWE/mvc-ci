\documentclass[oneside]{book}
%\documentclass{article}
\usepackage[utf8]{inputenc}
\usepackage[swedish]{babel}
\usepackage{biblatex}
\usepackage{csquotes}
\addbibresource{references.bib}

\setlength{\parskip}{1em}

\usepackage{hyperref}
\hypersetup{
    colorlinks=true,
    linkcolor=blue,
    filecolor=magenta,      
    urlcolor=blue,
}
 

\begin{document}

\begin{titlepage}
    \begin{center}
        \vspace*{1cm}
 
        \Huge
        \textbf{Min Kursrapport}
 
        \vspace{0.5cm}
        \LARGE
        DV1610\\
        Objektorienterade webbteknologier\\
        (mvc)
 
        \vspace{0.5cm}
        \LARGE
        Förnamn Efternamn\\
        min-epost@some.where\\

        \vspace{0.5cm}
        \LARGE
        v1.0.0\\
        Mars 16, 2021
 
        \vfill
 
        \vspace{2cm}
        \Large
        Läsperiod 4\\
        Våren 2021

        \vspace{0.5cm}
        \Large
        Institutionen för datavetenskap (DIDA)\\
        Blekinge Tekniska Högskola (BTH)\\
        Sverige
 
    \end{center}
\end{titlepage}

\tableofcontents



\chapter{Objektorientering}

\section{Berätta kort om dina förkunskaper och tidigare erfarenheter kring objektorientering. Kanske har du redan nu en uppfattning om det är bra eller ej?}

Mina förkunskaper kommer från python med kursen oopython där vi fick lära oss lite om det. Min uppfattning av det är att det kan vara svårt eller utmanande i början men 
sen släpper det lite när man liksom kommit in i det.

\section{Berätta kort om PHPs modell för klasser och objekt. Vilka är de grunder man behöver veta/förstå för att kunna komma igång och skapa sina första klasser?}

Vad jag känner speciellt man måste hålla koll på är hur man först bygger dem, efter det är det hur man "klassar" variabler
i klassen om de ska vara public eller private men sen också när man skriver i en funktion i en klass skriver man liksom inte
bara "variabel" utan man skriver "this"->"variabel" för att använda den variabeln i den klassen
där finns fler saker men de var huvusakerna jag glömde bort vid tillfällen eller inte tänkte på.

\section{Reflektera kort över den kodbas som användes till uppgiften, hur uppfattar du den?}
Jag uppfattar den som en bra bas där var mycket att läsa igenom så man hängde med men efter man lärt sig
det var det bara att utnyttja den tillfullo.

\section{Berätta om ditt spel från uppgiften. Hur löste du uppgiften, är du nöjd/missnöjd,
    vilken förbättringspotential ser du i koden/spelet,
    var uppgiften svårt/enkelt/utmanande, håller din kod god/hög kvalitet?}

Jag löste spelet igenom att använda en av klassera jag gjorde innan alltså diceHand för att kunna spela, Jag använde sessions för att
hålla koll på poängen men även vad man fått i tidigare kast. Jag använde post för att få input på hur många tärningar men även hur många
sidor spelaren ville ha. Efter det var det bara att bygga själva huvudelen med DiceHand och få allt att fungera utsendevis också. jag
skulle säga att jag är ganska nöjd med min slutdel där finns som sagt förbättringspotential så klart. En av de förbättringspotentialer
jag ser är att man kan fixa spelet så det ser bättre ut utsendemessigt på frontenden hade jag t.ex haft mer tid hade jag nog t.ex 
försökt lägga till så man ser i röd eller grön färg text som säger om man förlorat eller vunnit sen hade jag också försökt göra scoreborden
snyggare igenom att lägga till lite stil där också. uppgiften var enligt mig en ganska utmanande del för att vara första momentet i denna kurs
, den kunde kanske varit lite sbällare på en men sen samtidigt blickar jag lite in i framtiden och tänkre mig att där kommer ingen ta det lätt
med en utan där kommer det ställas liknande krav från start men troligtvis även mer så på så sätt var det en bra start. Min kod skulle
defenitivt kunnat blivit bättre med lite mer tid på händerna men som sagt har det varit många faktorer som kommit personligen som gjort det tufft
att hinna med mer.

\section{Med tanke på artikeln “PHP The Right Way”,
vilka delar in den finner du extra intressanta och
värdefulla? Är det några särskilda områden som du känner att du vill veta mer om?}

Jag läste igenom den lite snabbt då jag hade brottom och jag kan inte riktigt minnas
allt nu i efterhand men när jag tänker tillbaka på när jag läste det så kommer jag
inte ihåg någon speciell del jag tyckte var extra intressant eller värdefull eller
något område jag vill veta mer om.

\section{Vilken är din TIL för detta kmom?}

Min "TIL" för detta kursmoment är hur man skapar klasser i php.

\section{}



\chapter{Controller}

\section{Berätta på vilket sätt du drog nytta, eller inte, av att modellera din
lösning med flödesdiagram och psuedokod.
Använder du dig av top-down eller bottom-up när du planerar din kod?}

Jag kände personligen att jag inte drog någon direkt nytta av min psuedokod men mer av flödesdiagramet
där jag kunde följa hur man skulle gå till vida mer stegvis igenom spelet. Det blev väldigt ofta när jag skrev koden
att jag tappade lite vilken del jag skulle göra så att säga och då kunde jag kolla där snabbt hur spelet skulle gå till
och då var det bara att följa den efter det. Jag använde mig av top-down när jhag planerade min kod.

\section{Förklara kort de objektorienterade konstruktionerna arv, komposition, interface och trait och hur de används i PHP.}

Arv är då en klass som ärever från en "basklass" och utökar den för sina önskade mål. T.ex GameOf21 utökas från DiceHand i mitt fall
. komposition är då en klass som består av en annan klass. 
interface tilllåter dig att skapa kod som specificerar vilken metod en klass måste implementera utan att behöva
definera hur dessa metoder är implementerade. Enligt dbwebb defineras trait såhär: Man säger att en klass använder,
uses, ett trait. En klass kan använda flera traits. När klassen använder ett trait så blir dess kod som en del
av klassen. Man kan tänka sig att traitets kod kopieras in i klassen, det är en mental bild som fungerar.

\section{Berätta om ditt spel från uppgiften. Hur löste du uppgiften, är du nöjd/missnöjd,
vilken förbättringspotential ser du i koden/spelet,
var uppgiften svårt/enkelt/utmanande, håller din kod god/hög kvalitet?}

Mitt yatzy spel blev ganska bra, Hur jag gick tillvida för att lösa det var lite utmanande men
det gick till slut, jag deog mycket insperation från förra kursmomentet och kollade hur jag gjorde mycket där
från t.ex poäng till uppbyggare och mer. Sen fick jag öven bygga mycket nytt igen som t.ex poängräknaren
och även hur man väljer vilka man ska slå eller ej (tärningarna) och lite annat. Det jag ser som förbättringspotential
är för det mesta utsendemessigt då jag tänker på att man kanske kunde gjort det så man såg poängen som ett riktigt yatzy-
schema om man har sett en riktig lapp från spelet sen så klart kunde jag även gjort så man kunde försökt välja vilket man 
ville sätta poängen på så man inte behövde gå från 1 til 6 men även lägga till så man kunde få kanske kåk eller stege, tvåpar
osv. men då hade jag behövt myvket mer tid på mig för att hinan med något sånt. 
Uppgfiten skulle jag säga var lagom utmanande det var lite mer svårt för mig personligen att göra om hela routes och controller
delen personligen och fixa mitt gmae of 21 så det fungerade med det nya "systemet". Min kod skulle jag enligt mig säga håller
en ganska god kvalitet, hade jag hunnit så hade jag nog delat upp det i fler delar som hade jobbat tillsammans istället för att
göra allt på en enda "page" så att säga.

\section{Hur känner du för den kodstruktur som växer fram, tycker du det blev snyggare kod med modulerna router och request
och hur vi jobbade med controllers eller vad är din syn på det?}

Jag känner personligen nu efter detta kursmoment att det blir snyggare och bättre uppdelat men samtidigt också enklare
att liksom följa med, även om jag inte har erfarenheter av detta sedan innan så kunde jag endån med min relativa kunskap
följa med och se hur det gick till med hjälp av lektiosvideos.

\section{Vilken är din TIL för detta kmom?}

Min "TIL" för detta kursmoment är hur man kan bygga moduler med routes och request men även hur man jobbar med en controller.

\chapter{Enhetstestning}

Här skriver du din redovisningstext för detta kursmoment.



\chapter{Ramverk}

Här skriver du din redovisningstext för detta kursmoment.



\chapter{Autentisering}

Här skriver du din redovisningstext för detta kursmoment.



\chapter{To be defined}

Här skriver du din redovisningstext för detta kursmoment.



\chapter{Projekt \& Examination}

Här skriver du din redovisningstext för detta avslutande kursmoment.

\section{Projektet}

Här skriver du din redovisningstext rörande projektet.

\section{Avslutningsvis}

Här skriver du de avslutande orden om kursen.



\newpage
\printbibliography

\end{document}
